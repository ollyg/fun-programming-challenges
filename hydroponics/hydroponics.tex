\documentclass[11pt]{book}
%Gummi|063|=)
\author{Guy Edwards}
\date{19th April 2012}
\title{Hydroponics}
\begin{document}

\maketitle

\section{Foreword}

\paragraph{} [Note: this chapter is just a rough draft, if you want a more complete chapter take a look at the UAV folder tasks]

\paragraph{} This book aims to provide a set of problems, suitable for assisting learning any language that has a graphical library available to it. It's deliberately not language specific however your language should be able to cope with objects.

\paragraph{} The tasks start off at a reasonable level and are intended to have increasing difficulty. If learning programming for the first time you'll need a mentor to help guide you through background subjects that tasks cover, for instance when to use arrays to solve a problem. It might be possible to use the tasks as part of a taught course.

\paragraph{} I originally wrote this to help myself improve my programming. Any feedback on perceived missing steps or task difficulty jumps is welcome at guyjohnedwards@gmail.com


\paragraph{}

\tableofcontents

\chapter{Hydroponics}

For this set of problems you are a developer assisting a horticultural company in developing automated soil-less plant growing systems.

\clearpage

\section{Create Tests}

Construct a program that operates as a test harness for future development

\begin{enumerate}
\item write a function that takes a reading from the water level sensor
\item write a function that allows a test program to set the water level sensor reading
\item repeat for water pump control, ph level, nutrient concentration, nutrient pump, light level sensor, power usage, extractor fan rpm, extractor fan status (on/off), air temperature and air humidity
\item write an interface so that a testing operator can set the variables as desired
\end{enumerate}

\clearpage

\section{Closed loops}

In this task we're creating basic automatic feedback

\begin{itemize}
\item when the water level is low, more is pumped in from a reservoir
\item when the humidity is too high an extractor fan turns on
\item when the air temperature is too high an extractor fan turns on
\item [more]
\end{itemize}

\clearpage

\section{More control}

Construct a program such that

\begin{itemize}
\item the operator can see the outputs from the various sensors
\item the operator can adjust settings of pumps, valves and switches
\item this operations interface is separate to the testing interface
\end{itemize}

\clearpage

\section{Refinement}


[to follow]



\end{document}