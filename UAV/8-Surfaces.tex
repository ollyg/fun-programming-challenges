\documentclass[11pt]{book}
%Gummi|062|=)
\title{\textbf{Programming Tasks}}
\author{Guy Edwards}
\date{30th March 2012}
\begin{document}

\section{Control Surfaces}

\subsection{Problem Definition}

\paragraph{} At some point our simulation will take over the hardware of the UAV itself. We can take actual real flight, but the control software needs to activate the sub systems on the UAV to achieve flight.

\paragraph{} For now the subroutines don't need to activate any motors or similar, but are in place so that they can be modified to do so.

\subsection{Requirements}

\begin{enumerate}
\item Write a rudder control subroutine, affecting yaw
\item Write a aileron control subroutine, affecting pitch and roll
\item Write a throttle control subroutine
\item Write a landing gear control subroutine
\item Write a camera control subroutine
\item Ensure simulated movement behaviour has a time value (a 90 degree roll takes time to complete)
\item A heavier plane (full fuel load) should behave slightly differently
\item Log all control surface moment
\item Change the game world to operate at fps, not iteration sleep intervals
\end{enumerate}

\subsection{Why Are We Doing This?}

\paragraph{} With some extra future work in the following projects, this will help to allow genuine flight control of a working model. It can also be used to allow a future detailed working graphical representation of the aircraft.

\paragraph{} It's directly relevant to flight simulation games, to automated control of flying vehicles and as an example of control of mechanical objects via software.

\subsection{Bonus Tasks}

\begin{itemize}
\item Can you display the control surface state in the GUI?
\item Does the game world appear smooth? You might want to make the roll and pitch fairly slow so that you can see the UAV activating the flight surfaces and tracking the movements of the target.
\end{itemize}

\end{document}
