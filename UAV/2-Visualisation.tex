\documentclass[11pt]{book}
%Gummi|062|=)
\title{\textbf{Programming Tasks}}
\author{Guy Edwards}
\date{30th March 2012}
\begin{document}

\section{Project - Visualisation}

\subsection{Problem Definition}

\paragraph{} Your divisional manager who is funding your UAV project doesn't
understand programming or the text output from the first project and wants to
see visually what is going on in the simulation so that the manager knows the
project money funding your position is producing results.

\paragraph{} You are going to write a Graphical User Interface (GUI) so that
the manager can watch the UAV movement on a desktop machine.

\subsection{Requirements}

\begin{enumerate}
\item Write a graphical interface that displays the 100x100 map grid.
\item Show a dot on the grid for the position of the UAV.
\item Show a dot on the grid for the position of the destination.
\item Create a "Start" button that, when clicked, starts the simulation running.
\item When the simulation finishes, the application should stay open and show the last positions.
\item When the simulation ends, clicking the start button should start another simulated run.
\item Don't make behavioural changes to the underlying simulation in this project.
\item Keep it as one program, avoid writing a separate GUI program that runs from the information in the text output of the first.
\end{enumerate}

\subsection{Why are we doing this?}

\paragraph{} This is our first application using a graphical toolkit. The aim
is gentle familiarisation with the chosen graphical toolkit, which should be
tricky enough at first. The following tasks are move adventurous so don't leap
ahead.

\paragraph{} There are deliberately no behaviour changes introduced to our
previous code, we're visualising what is already happening.

\subsection{Bonus tasks}

\paragraph{} When watching a simulation, your important divisional manager
likes to interrupt and ask what exactly is going on.

\begin{itemize}
\item Add a "Pause" button so you can pause the simulation while you explain what's occurring on screen.
\item On pausing make the button text change to "Resume" so it's clear that the simulation is paused.
\item When a simulation is actively running, disable the start button to avoid a user getting confused.
\item Add a sub window in the program which shows the text output from the program as it is produced to help you explain to a spectator what the program has decided and why.
\end{itemize}

\clearpage

\end{document}
