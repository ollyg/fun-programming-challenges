\documentclass[11pt]{book}
%Gummi|062|=)
\title{\textbf{Programming Tasks}}
\author{Guy Edwards}
\date{30th March 2012}
\begin{document}

\section{Line of Sight}

\subsection{Problem Definition}

\paragraph{} The camera on the UAV can't picture everything within a given range, it has a given envelope in which it can picture, for example it's placed on the nose of the UAV and can photograph in a half-sphere with useful definition to a specific range (e.g. 200metres maximum). We need to adjust our control software to cope with this.

\subsection{Requirements}

\begin{enumerate}
\item Ensure the UAV has a orientation property
     \begin{enumerate}
         \item a front
         \item pitch
         \item yaw
         \item roll
     \end{enumerate}
\item Add properties for the UAV cameras area/range envelope
     \begin{enumerate}
         \item range
         \item sweep
     \end{enumerate}
\item Update the GUI to show the UAV orientation, you could use a simple 3 axis cross or something mode advanced such as a wireframe model
\item Update the GUI to show the UAV camera area, perhaps as a translucent cone
\end{enumerate}

\subsection{Why Are We Doing This?}

\paragraph{} The line of sight is used in games (can the alien/shark/enemy see you?) but also for any industrial application where an appliance (camera, blowtorch, electronic sensor) has to be within a certain distance of an object).

\paragraph{} The changes both to the program and the GUI are fairly involved, so this is not a small amount of work.

\subsection{Bonus Tasks}

\begin{itemize}
\item Make a button in the GUI that causes a gust of wind that flips the UAV upside down. Does the UAV self right itself? Can you create functionality so that is does?
\end{itemize}

\end{document}
